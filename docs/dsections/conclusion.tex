\section{Conclusion}

This work now provides a four-model spectral computation baseline in C++20: iterative radix-2 FFT, recursive radix-2 FFT, split-radix FFT, and a direct DFT reference. The implementations are grounded in explicit Fourier sign conventions, deterministic inverse normalization, and strict automated verification against reference behavior.

Benchmark results confirm the expected asymptotic gap between FFT models and direct DFT. They also show that split-radix variants deliver meaningful runtime and throughput improvements over recursive radix-2 in this environment, but do not surpass the iterative radix-2 implementation for the tested sizes. This behavior is consistent with practical trade-offs: split-radix reduces arithmetic operations, yet memory-access patterns, data movement, and traversal overhead remain relevant on modern desktop CPUs.

The resulting framework supports future extensions such as mixed-radix decomposition, real-input specialization, SIMD acceleration, planner-based optimization \cite{frigo2005}, and embedded profiling, while preserving a mathematically auditable core.
