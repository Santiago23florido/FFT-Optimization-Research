\section{Conclusion}

This work provides a correctness-verified FFT framework in C++20 with two complementary benchmark layers. The primary layer compares algorithmic families (iterative radix-2, mixed radix (4/2), recursive radix-2, split-radix, and direct DFT). The secondary layer isolates memory-layout effects (AoS versus SoA) for the two strongest iterative families.

The primary study confirms the expected complexity separation: FFT factorizations preserve scalable behavior while direct DFT becomes rapidly impractical as \(N\) grows. The layout study then shows how data organization alone can shift runtime for a fixed transform definition and fixed benchmark protocol.

Overall, the framework now supports both algorithm-level and memory-level optimization analysis under a single reproducible pipeline. This structure is suitable for subsequent work on explicit SIMD kernels, planner-based execution strategies \cite{frigo2005}, and embedded performance portability while preserving mathematical traceability and automated verification.
