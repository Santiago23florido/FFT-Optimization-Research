\section{Algorithm and Implementation Mapping}

The codebase evaluates multiple transform models under one interface, including iterative radix-2 FFT, iterative mixed-radix (4/2) FFT, recursive radix-2 FFT, split-radix FFT, direct \(O(N^2)\) DFT reference, and new SoA layout counterparts for radix-2 and mixed-radix. This section details the mapping used by the radix-2 iterative baseline, while mixed-radix and split-radix decompositions are detailed in Sections \ref{sec:mixed_radix_fft} and \ref{sec:split_radix_fft}, respectively.

\subsection{Iterative In-Place Kernel}

The implementation processes stages with block lengths \(L=2,4,\ldots,N\). For each block start index \(i\) and local offset \(j\in[0,L/2-1]\):
\begin{align}
u &= x[i+j],\\
v &= x[i+j+L/2]\,W_L^j,
\end{align}
then updates
\begin{align}
x[i+j] &\leftarrow u+v,\\
x[i+j+L/2] &\leftarrow u-v.
\end{align}
This mapping is a direct implementation of the butterfly equations.

\subsection{Twiddle Generation Strategy}

At each stage, the principal stage twiddle \(W_L=e^{\pm j2\pi/L}\) is computed once. Per-butterfly twiddles are obtained by recurrence \(w\leftarrow wW_L\) within each block. This avoids storing global twiddle tables while preserving deterministic behavior and numerical clarity.

\subsection{Inverse Transform}

The inverse transform reuses the same kernel with opposite sign in the exponential and applies a final scaling by \(1/N\). This ensures consistency with \eqref{eq:idft} and supports stable round-trip reconstruction.

\subsection{Input Validation}

Because the chosen factorization is radix-2, valid sizes must satisfy \(N\neq 0\) and \(N\) power-of-two. Invalid sizes throw \texttt{std::invalid\_argument} with an explicit message, preventing undefined spectral behavior.

\subsection{Implementation Notes}

The code is written in C++20 with only the standard library, using \texttt{std::complex<double>} for numerical representation. The design goal is not architecture-specific peak throughput but reference-quality correctness that remains portable across Linux and Windows toolchains.
