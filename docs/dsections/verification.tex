\section{Verification Methodology}

\subsection{Automated Test Set}

Verification is executed through CTest and a self-contained test executable. The FFT-oriented checks are applied to all power-of-two FFT models, including AoS and SoA variants of iterative radix-2 and iterative mixed-radix (4/2), plus recursive radix-2 and split-radix. The direct DFT model is validated through dedicated checks. The suite includes:
\begin{itemize}
\item \textbf{Round-trip stability:} for powers of two from \(N=2\) to \(N=4096\), verify \(\mathrm{IFFT}(\mathrm{FFT}(x))\approx x\) using random complex vectors with fixed seed.
\item \textbf{Reference agreement:} for \(N\leq 256\), compare each FFT bin against the direct \(O(N^2)\) DFT.
\item \textbf{Layout consistency:} verify that SoA outputs match their AoS counterparts for the same input and ordering convention.
\item \textbf{Pure-tone localization:} for \(x[n]=e^{j2\pi kn/N}\), verify dominant energy at bin \(k\) and near-zero leakage elsewhere.
\item \textbf{Parseval consistency:} verify
\begin{equation}
\sum_n |x[n]|^2 \approx \frac{1}{N}\sum_k |X[k]|^2,
\end{equation}
consistent with the chosen normalization.
\item \textbf{Input constraints:} verify rejection of non-power-of-two sizes.
\end{itemize}

\subsection{Tolerances and Rationale}

Double-precision tolerances are set to \(10^{-10}\) for round-trip relative \(L_2\) error, bin-wise FFT-vs-DFT absolute error, and Parseval relative mismatch. A \(10^{-8}\) leakage bound is used for off-peak bins in pure-tone tests. These values reflect expected floating-point accumulation and trigonometric evaluation error under IEEE-754 arithmetic \cite{ieee754}.

\subsection{Expected Spectral Behavior}

A single-tone complex exponential is expected to produce one dominant peak at the target bin \(k\). A real sine wave is expected to produce two symmetric dominant peaks at \(k\) and \(N-k\), consistent with conjugate symmetry for real-valued time signals.
