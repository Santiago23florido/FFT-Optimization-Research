\section{Benchmark Results and Comparison}

This section reports the primary algorithmic comparison study. The evaluated models are:
\texttt{radix2\_aos}, \texttt{mixed42\_aos}, \texttt{radix2\_recursive}, \texttt{split\_radix}, and \texttt{direct\_dft}.  
The benchmark uses \(N\in\{64,128,256,512,1024,2048,4096\}\), 5 warmup iterations, 40 measured iterations, and a fixed seed.

The reported figures summarize:
\begin{itemize}
\item transform length \(N\);
\item average runtime per transform;
\item throughput in processed samples per second.
\item Additional descriptors are also measured in the pipeline: median runtime, minimum and maximum runtime, runtime dispersion, high-percentile latency, and normalized costs per sample and per \(N\log_2 N\).
\end{itemize}

\begin{figure}[t]
\centering
\includegraphics[width=\linewidth]{figures/mean_runtime.png}
\caption{Average execution time as a function of signal length \(N\) for the primary algorithmic study.}
\label{fig:mean_runtime}
\end{figure}

\begin{figure}[t]
\centering
\includegraphics[width=\linewidth]{figures/throughput.png}
\caption{Throughput as a function of signal length \(N\) for the primary algorithmic study.}
\label{fig:throughput}
\end{figure}

Figure \ref{fig:mean_runtime} shows the expected asymptotic separation between FFT factorizations and direct DFT. The direct \(O(N^2)\) model becomes rapidly dominant in runtime as \(N\) increases, while all FFT variants preserve the \(O(N\log_2 N)\) trend.

Figure \ref{fig:throughput} presents the same behavior in throughput units. Iterative implementations remain the strongest practical baseline in this environment, and recursive decompositions remain useful as algorithmic references but with lower effective throughput due to call and traversal overhead.

This primary study provides the baseline against which the dedicated memory-layout study in Section \ref{sec:aos_soa} is interpreted.
