\section{Mathematical Foundations}

\subsection{DFT and IDFT Definitions}

For a length-\(N\) complex sequence \(x[n]\), the forward transform is
\begin{equation}
X[k] = \sum_{n=0}^{N-1} x[n] e^{-j2\pi kn/N},\quad k=0,\ldots,N-1,
\label{eq:dft}
\end{equation}
and the inverse transform is
\begin{equation}
x[n] = \frac{1}{N}\sum_{k=0}^{N-1} X[k] e^{+j2\pi kn/N},\quad n=0,\ldots,N-1.
\label{eq:idft}
\end{equation}
The implementation follows this sign convention exactly and applies normalization only in the inverse path.

\subsection{Roots of Unity and Notation}

Define
\begin{equation}
W_N = e^{-j2\pi/N},
\end{equation}
so that \(W_N^{kn}=e^{-j2\pi kn/N}\). Equation \eqref{eq:dft} becomes
\begin{equation}
X[k] = \sum_{n=0}^{N-1} x[n] W_N^{kn}.
\end{equation}
This notation enables compact derivation of radix-2 factorization.

\subsection{Radix-2 Cooley-Tukey Decomposition}

Assume \(N=2^m\). Split the sequence into even and odd indices:
\begin{equation}
X[k]=\sum_{r=0}^{N/2-1}x[2r]W_N^{k(2r)}+\sum_{r=0}^{N/2-1}x[2r+1]W_N^{k(2r+1)}.
\end{equation}
Using \(W_N^2=W_{N/2}\), define
\begin{equation}
E[k]=\sum_{r=0}^{N/2-1}x[2r]W_{N/2}^{kr},\quad
O[k]=\sum_{r=0}^{N/2-1}x[2r+1]W_{N/2}^{kr},
\end{equation}
then
\begin{align}
X[k] &= E[k] + W_N^k O[k],\label{eq:butterfly_a}\\
X[k+N/2] &= E[k] - W_N^k O[k].\label{eq:butterfly_b}
\end{align}
Equations \eqref{eq:butterfly_a} and \eqref{eq:butterfly_b} are the radix-2 butterfly relations.

\subsection{Bit-Reversal Permutation}

In an iterative decimation-in-time schedule, input indices are permuted by reversing their \(m\)-bit binary representation before stage-wise butterflies are applied. This permutation maps recursive subproblem order into contiguous in-place blocks. Without it, stage-local butterfly connectivity is violated and final bins are scrambled.

\subsection{Complexity}

A direct DFT computes \(N\) outputs, each with \(N\) terms, yielding \(\Theta(N^2)\). The radix-2 FFT performs \(\log_2 N\) stages with \(\Theta(N)\) work per stage, yielding \(\Theta(N\log_2 N)\). This is the computational foundation for real-time spectral systems.
