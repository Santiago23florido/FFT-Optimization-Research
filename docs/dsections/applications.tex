\section{Scientific and Engineering Relevance}

FFT is foundational in projects where information is encoded in spectral structure rather than raw samples. In digital communications, orthogonal frequency-division multiplexing (OFDM) modulators and demodulators rely on FFT/IFFT pairs for subcarrier-domain processing and channel equalization. In radar and sonar, FFT stages support range-Doppler processing and clutter separation. In biomedical instrumentation, electroencephalography and electrocardiography pipelines use spectral features for diagnosis and anomaly detection.

In imaging and inverse problems, FFT-based convolution accelerates iterative solvers for magnetic resonance reconstruction, computational microscopy, and deblurring. In mechanical and civil monitoring, vibration spectra expose resonance modes and early fault signatures in rotating machinery and structures. In power systems, harmonic analysis depends on FFT stability to quantify distortion and nonstationary events.

For machine-learning systems, FFT-derived features are frequently used in audio classification, activity recognition, and predictive maintenance. In these data-centric workflows, a well-verified FFT implementation reduces silent label noise caused by spectral misalignment and improves reproducibility across hardware and operating systems.

Therefore, FFT is not only an algorithmic acceleration; it is a scientific instrumentation primitive. Correctness, traceable conventions, and controlled numerical error are mandatory for high-impact engineering projects.
